To support the key-value cache, each node must implement some form of indexing
service. Each indexing service must support a number of tasks such as
\emph{insertion}, \emph{deletion}, \emph{retrieval}, \emph{eviction} and
\emph{migration}. We consider three distinct but common indexing schemes:
\bptrees\cite{btree,bplustree}, Extendible Hashing\cite{ullman}, and
Counting Bloom Filters\cite{countingbloom1,countingbloom2}.

In order to operate in the elastic environment of our cache, when a node
overflows it must be capable of migrating a subset of its data records to
another node either preexisting or freshly allocated. Each of these schemes
have inherent differences in their structure and operation and, as such, are
compelling candidates for extension into the elastic makeup of \emph{Auspice}.

The remainder of this chapter will present background on each of the indexing
structures, describe the implementation of their migration mechanisms, and
provide an experimental analysis of their performance benchmarks.

\section{\bptrees} % (fold)
\label{sec:b_trees}
B-Trees and \bptrees are used in many of today's systems. The \bptree is a
multilevel indexing scheme that automatically adjusts the number of levels
depending upon the file size and stores all of its records in the leaf nodes.
Each record is stored in ascending order from left to right and each leaf node
is linked to the next and previous nodes. In this way, its design is
specifically crafted to accelerate queries over a range of
values\cite{navathe,ullman}. In terms of retrieval, \bptrees are balanced data
structures, where all paths from the root to any leaf have the same length
(similar to binary trees, with approximately $log_2 n$ depth).

% TODO: Figure for B+-Tree (Wikipedia has a good example, also on the
% whiteboard still), structural explanation. This section is also cribbed
% heavily from IJNGC paper, look at rewrite.

% section B_Trees (end)

\section{Extendible Hashing} % (fold)
\label{sec:extendible_hashing}

% section extendible_hashing (end)

\section{Bloom Filters} % (fold)
\label{sec:bloom_filters}

% section bloom_filters (end)

\section{Experiments} % (fold)
\label{sec:experiments_indexing}

% section experiments_indexing (end)

\section{Results} % (fold)
\label{sec:results_indexing}

% section results_indexing (end)
