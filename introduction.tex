In recent years, the Internet community has innovated extraordinarily well.
Platforms such as \emph{Facebook}, \emph{Twitter}, and \emph{Amazon} have
pushed the boundaries of web technology, demonstrating on a large scale, the
impact and viability of social media and e-commerce. Now more than ever,
businesses and startups are flocking to the web to offer their services.
However, with the web comes a set of new and unique challenges. The sheer
monolithic size of one's audience, the tremendous amount of data associated
with the service, the capricious nature of many web-users, and the rapid pace
at which competition develops name only a few such difficulties.

One of the more interesting ways of addressing these problems of storage,
computation, and stability is to utilize the Cloud. Within the cloud,
developers are given access to an ``infinite'' realm of storage and computation
which they can provision and utilize according to their needs. This concept of
Infrastructure as a Service (IaaS) allows even the smallest businesses to
provision the necessary computational resources without the overhead involved
in purchasing, configuring, and maintaining a private infrastructure. In
essence, computation becomes a utility similar to electricity and water
supplies, rather than a fixed value to be allocated across a set of problems,
and users are charged in accordance with the amount of resources they consume.

Among cloud providers, Amazon's Web Services (AWS) has risen to a seat of
prominence. In AWS, one of the most important and driving aspects is the notion
of \emph{elasticity}. Put another way: on demand scaling of resources. Though
AWS is composed of many different services the most notable demonstration of
this concept is the Amazon Elastic Compute Cloud (EC2). Users of EC2 can
instantly scale, both up \emph{and} down, the amount of resources allocated to
a task, neatly avoiding problems of both under- and over-provisioning.
Similarly, Amazon's Simple Storage Service (S3) provides access to cloud-backed
storage space, allowing users to offload data from their servers to Amazon's in
a secure and stable manner.

By utilizing a combination of EC2 and S3, web service providers can scale their
resource usage in accordance to their needs. When computational speed becomes
an issue, developers can add another compute node to help service their users'
requirements, helping to alleviate unexpected downtime or slowness -- two major
players in determining whether a userbase will stick with a product or if they
will move to another. Likewise, when the amount of data being generated becomes
too large or cumbersome, developers and technicians can move that data into the
cloud, potentially freeing them from having to purchase another data server.

For a simple example, consider the requirements of a microblogging service
(which we will call \emph{Aflutter}). Much of the time, Aflutter receives a
steady flow of traffic and thus their designers and developers have planned
around it, purchasing an infrastructure suitable for those needs. However, as
events transpire throughout the world, users are driven to communicate about
them rapidly, informing one another, providing commentary, or sometimes simply
to make jokes. Given that these traffic spikes occur in response to events in
the world around us, they are supremely difficult, if not impossible to
anticipate. With the cloud, expansion and accomodation can occur responsively
and the engineers in charge of keeping Aflutter running smoothly can allocate,
on the fly, additional computational power to help cope with this surge of
usage. As the event dies down and demand returns more towards normal levels,
these extra nodes can be freed up and the costs associated with their usage
cease to accumulate.

The emergence and realization of this elastic model has been propitious --
elasticity can and has been beneficial for many applications, but with it comes
its own assortment of challenges for developers. Among these challenges is the
difficulty of managing data in a highly dynamic environment. In this paper, we
explore the issues associated with creating and managing an elastic
\emph{key-value} storage system.

\emph{Auspice}, the product of this exploration, was motivated by challenges in
accelerating \emph{service-oriented} computations in the cloud. In our earlier
works, the key-value cache was used to store the intermediate results of web
services which were then used in the process of completing other computations.
Similar to our previous example, this type of cache benefits from elasticity as
periods of intensive querying can be responded to through the self-managed
expansion of resources.

During the creation of \emph{Auspice}, we also identified a number of areas for
improvement. The two primary areas we recognized were in the macro and micro
levels, that is, the management of the cache as a whole and the management of
each individual cache node. This research is focused additionally on the micro
level. In particular, this paper focuses on mechanisms for managing data
placement within the cache and prioritizing the retrieval of high-value
elements.

\section{Research Contributions} % (fold)
\label{sec:Research_Contributions}
The main goal of this research is to explore and evaluate the design and
optimizations of the individual cache nodes. To do that, we implement an
elastic key-value store suitable for the cloud environment, including efficient
algorithms for inserting, evicting, deleting, searching, and migrating data.
The main contributions of our work are as follows:
\begin{itemize}
  \item Exploring three different data indexing schemes: B$^+$-trees, Extendible
    Hashing, and Bloom filters.
  \item Providing mechanisms for automated cache scaling.
  \item Evaluating multiple approaches for data eviction ranging from First-In
    First-Out (FIFO) to our own custom approach.
  \item Creating a hierarchy of data storage mediums that reflect the
    importance of the data stored within.
  \item Mechanisms for dynamically placing data within the hierarchy.
\end{itemize}

% section Research Contributions (end)

\section{Thesis Overview} % (fold)
\label{sec:Thesis_Overview}
The ultimate goal of this thesis is to describe the various mechanisms
implemented to improve the performance of each cache node. The following
chapters provide additional background and the details involved in this
pursuit. Chapter~\ref{chap:Background} provides background for the creation of
\emph{Auspice}, our key-value store, as well as information about Amazon Web
Services (AWS). Chapter~\ref{chap:Indexing_Strategies} includes an analysis of
three data indexing schemes and the tradeoffs between them.
Chapter~\ref{chap:Eviction_Strategies} evaluates different data eviction
strategies and their impact on cache performance in situations where unlimited
storage is infeasible. Chapter~\ref{chap:Storage_Mediums} establishes a
hierarchy of data storage mediums as well as mechanisms for evaluating the
``importance'' of data units, placing them within the hierarchy, and evicting
them from the cache. Finally, Chapter~\ref{chap:Conclusion} concludes the
thesis and provides a discussion of future works.

% section Thesis Overview (end)
