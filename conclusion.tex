\section{Conclusion} % (fold)
\label{sec:conclusion}
Back in Chapter~\ref{chap:Introduction} we motivated the development of our
research on the concept of \emph{elasticity}. The idea that a system need not
be defined rigidly in terms of its computational and storage capacity.  That
Amazon's Elastic Compute Cloud (EC2) and other \emph{Infrastructure as a
Service} providers open up the possibility for soft systems that scale with the
demand of their workload. We began with a simple example of a microblogging
webservice that we called \emph{Aflutter}, and we described the ways in which
this elasticity can help to smooth out spikes in service demand without
over-provisioning their computational backbone.

Since then, we have introduced and explored the performance implications
inherent in a cloud-based \emph{elastic cache} for both computational and
data-intensive applications. In previous
works\cite{chiu_ijngc11,chiu_ccgrid11}, we investigated different options for
indexing schemes in our individual cache nodes and settled on utilizing
\bptrees in the general case. In this paper, we continued our studies with the
elastic cache by adding new features. In Chapter~\ref{chap:Hybrid_Cache} we
implemented a new form of cache node, expanding upon the previous design,
adding a layer of persistent storage. This allowed us to create a form of
elastic cache that did not \emph{always} have to grow. The system was allowed
to evict data from memory into a layer of persistent storage, reducing the rate
at which the elastic cache grew in number of nodes. Additionally, our node
demonstrated higher hit rates and lower average request times.

By allowing our nodes to evict data from memory into the persistent layer, we
introduced two concepts deserving of scrutiny: \emph{eviction schemes} and the
notion of a \emph{storage hierarchy}. In Chapter~\ref{chap:Eviction_Strategies}
we explored the former, contrasting the performances of both Least Recently
Used (LRU) and First-In First-Out (FIFO) strategies. There our results were
somewhat inconclusive, showing that the performance differences between LRU and
FIFO were not very large. We did note a slight preference for LRU in the
general case and especially early on into the cache's lifespan. For this reason
we opted to utilize the LRU queue as our baseline eviction scheme.

Finally, Chapter~\ref{chap:Storage_Mediums} explored the creation of a
\emph{storage hierarchy}, utilizing the full range of a node's storage
capacity. As a result of this expansion, we also continued to develop our
\emph{eviction schemes}. These mutated, as a result of the storage hierarchy,
becoming \emph{data placement schemes}. These schemes determine which layer of
our memory-disk-cloud hierarchy that each key-value pair is to be stored on.
Here we created a \emph{Cost Aware} (CA) placement strategy that resulted in a
much larger number of hits-in-memory over the baseline LRU strategy.
Subsequently, average response time improved yet again.

% section conclusion (end)

\section{Future Work} % (fold)
\label{sec:future_work}
Truthfully, we have only scratched the surface of what is possible in an
elastic cache. In our creation of a Cost Aware data placement strategy we
recognized many areas for improvement and further study. From the perspective
of the scoring algorithm alone, there is opportunity for the creation of a much
more nuanced function. While we pulled inspiration from two common ranking
algorithms in information retrieval\cite{tfidf,bm25}, there is an exceptionally
large amount of research in the field that could be further adapted to this
purpose.

Additionally, the mechanisms by which we handle the rearrangement and
re-scoring of data are deserving of greater inspection.
Algorithm~\ref{alg:ca_data_placement} is a very straightforward and simple
approach to rearranging our data. An approach that manages this in a more
elegant and time-sensitive manner would almost certainly be necessary when
workloads are so intense as to stress the limits of the cache.

Finally, there is much interest in exploring a comparison between our elastic
cache and ones that already exist. For example, running benchmarks against
Amazon's ElastiCache\cite{amazonElastiCache} would be a solid test of our
design and implementation. Unfortunately, due to time constraints, we were
never able to run such experiments.

% section future_work (end)
