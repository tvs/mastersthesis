As the Internet matures and data-intensive services become the norm, the
concept of cloud computing is becoming increasingly appealing. The cloud
provides near instantaneous, paid access to an infinite realm of storage and
computing without the tremendous cost of designing and purchasing such a
framework. By alleviating this upfront cost, cloud providers, such as Amazon,
rent their infrastructure out to users, allowing them complete control over
both compute and storage reources. One of the primary driving notions is the
concept of \emph{elasticity}: as demand fluctuates, users are able to scale up
or down their resource allocation in real-time.

In previous research, we focused on the challenges of data management in Cloud
environments. Specifically, we considered an elastic \emph{key-value} storage
system being used to cache intermediate results in a service-oriented system
and then accelerate future queries through reuse. One of the areas we
identified for improvement is in placing the data on the individual cloud
nodes: should data be in memory, on disk, or in a more persistent storage
location? To address this, we developed a scoring algorithm for analyzing each
key-value pair. We take into account such things as the \emph{total memory
size} of the instance, the \emph{frequency} with which the key is ``hit'', the
\emph{retrieval-time} for the data associated with the key and the overall
amount of time since the data was last accessed. This score is then used to
promote or demote each pair into the appropriate storage area.
